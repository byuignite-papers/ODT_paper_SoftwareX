%% 
%% Copyright 2007, 2008, 2009 Elsevier Ltd
%% 
%% This file is part of the 'Elsarticle Bundle'.
%% ---------------------------------------------
%% 
%% It may be distributed under the conditions of the LaTeX Project Public
%% License, either version 1.2 of this license or (at your option) any
%% later version.  The latest version of this license is in
%%    http://www.latex-project.org/lppl.txt
%% and version 1.2 or later is part of all distributions of LaTeX
%% version 1999/12/01 or later.
%% 
%% The list of all files belonging to the 'Elsarticle Bundle' is
%% given in the file `manifest.txt'.
%% 

%% Template article for Elsevier's document class `elsarticle'
%% with numbered style bibliographic references
%% SP 2008/03/01

\documentclass[preprint,12pt, a4paper]{elsarticle}

%% Use the option review to obtain double line spacing
%% \documentclass[authoryear,preprint,review,12pt]{elsarticle}

%% For including figures, graphicx.sty has been loaded in
%% elsarticle.cls. If you prefer to use the old commands
%% please give \usepackage{epsfig}

%% The amssymb package provides various useful mathematical symbols
\usepackage{amssymb}
%% The amsthm package provides extended theorem environments
%% \usepackage{amsthm}

%% The lineno packages adds line numbers. Start line numbering with
%% \begin{linenumbers}, end it with \end{linenumbers}. Or switch it on
%% for the whole article with \linenumbers.
\usepackage{lineno}

\usepackage{float}
\restylefloat{table}

\journal{SoftwareX}

\begin{document}

%%%%%%%%%%%%%%%%%%%%%%%%%%%%%%%%%%%%%%%%%%%%%%%%%%%%%%%%%%%%%%%%%%%%%%%%%%%%%%%
\begin{frontmatter}

%% Title, authors and addresses

%% use the tnoteref command within \title for footnotes;
%% use the tnotetext command for theassociated footnote;
%% use the fnref command within \author or \address for footnotes;
%% use the fntext command for theassociated footnote;
%% use the corref command within \author for corresponding author footnotes;
%% use the cortext command for theassociated footnote;
%% use the ead command for the email address,
%% and the form \ead[url] for the home page:
%% \title{Title\tnoteref{label1}}
%% \tnotetext[label1]{}
%% \author{Name\corref{cor1}\fnref{label2}}
%% \ead{email address}
%% \ead[url]{home page}
%% \fntext[label2]{}
%% \cortext[cor1]{}
%% \address{Address\fnref{label3}}
%% \fntext[label3]{}

\title{One-dimensional turbulence (ODT): computationally efficient modeling and simulation of turbulent  flows}

%% use optional labels to link authors explicitly to addresses:
%% \author[label1,label2]{}
%% \address[label1]{}
%% \address[label2]{}

%\renewcommand{\thefootnote}{\fnsymbol{footnote}}
\author{Victoria B. Stephens}
\author{David O. Lignell\corref{cor1}}

\cortext[cor1]{Corresponding author. \ead{davidlignell@byu.edu}}

\address{Chemical Engineering Department, Brigham Young University, Provo, UT 84602, USA}

\begin{abstract}
Write this last. About 100 words. 
\end{abstract}

\begin{keyword}
%% keywords here, in the form: keyword \sep keyword
turbulence \sep reacting flows \sep one-dimensional turbulence

%% PACS codes here, in the form: \PACS code \sep code

%% MSC codes here, in the form: \MSC code \sep code
%% or \MSC[2008] code \sep code (2000 is the default)

\end{keyword}

\end{frontmatter}

%%%%%%%%%%%%%%%%%%%%%%%%%%%%%%%%%%%%%%%%%%%%%%%%%%%%%%%%%%%%%%%%%%%%%%%%%%%%%%%
\section*{Code Metadata}
\label{metadata}

\begin{table}[H]
\begin{tabular}{|l|p{6.5cm}|p{6.5cm}|}
\hline
\textbf{Nr.} & \textbf{Code metadata description} & \textbf{Please fill in this column} \\
\hline
C1 & Current code version & 1.0 \\
\hline
C2 & Permanent link to code/repository used for this code version & $github.com/BYUignite/ODT$ \\
\hline
C3 & Code Ocean compute capsule & N/A\\
\hline
C4 & Legal Code License   & MIT \\
\hline
C5 & Code versioning system used & Git \\
\hline
C6 & Software code languages, tools, and services used & C++, Python 3.x, Yaml,  \\
\hline
C7 & Compilation requirements, operating environments \& dependencies & CMake 3.12+, Cantera, Git, Doxygen (optional) \\
\hline
C8 & If available Link to developer documentation/manual & N/A \\
\hline
C9 & Support email for questions & davidlignellbyu.edu \\
\hline
\end{tabular}
\caption{Code metadata (mandatory)}
\end{table}


\linenumbers

%The permanent link to code/repository or the zip archive should include the following requirements: 
%\begin{itemize}
%	\item README.txt and LICENSE.txt.
%	\item Source code in a src/ directory, not the root of the repository.
%	\item TO DO: Tag corresponding with the version of the software that is reviewed.
%	\item TO DO: Documentation in the repository in a docs/ directory, and/or READMEs, as appropriate.
%\end{itemize}

%%%%%%%%%%%%%%%%%%%%%%%%%%%%%%%%%%%%%%%%%%%%%%%%%%%%%%%%%%%%%%%%%%%%%%%%%%%%%%%
\section{Motivation and significance}
\label{sec:motivation}

Turbulent flows characterize the vast majority of fluid flows in practical engineering applications, and simulations of turbulent flows provide researchers with valuable insights into complex systems, particularly reacting turbulent flows such as combustion processes. Turbulence is a complex phemonenon that affects the full range of a flow's length and time scales. As a result, resolving the entire flow field by numerically solving the Navier-Stokes equations of fluid flow, as is done in direct numerical simulations (DNS), requires substantial computational resources. DNS is a powerful research tool, but its high computational cost makes it intractable for simulating most practical engineering flows. In order to achieve numerical solutions to practical flow problems, researchers can use alternative frameworks that model turbulence rather than resolving it directly.

Large-eddy simulations (LES) address the problem of wide-ranging length and time scales by combining  direct resolution of grid-scale quantities, as in DNS, with subgrid modeling of smaller turbulence structures. The more complex the flow, the more modeling is required; for example, a jet flame simulation might require subgrid modeling for the combustion chemistry, radiative heat transfer, or soot chemistry in addition to turbulence structures, all of which form a tightly coupled system in which each model interacts heavily with the others. While subgrid modeling makes LES more computationally affordable than DNS, it can introduce empiricism into simulations, which can lead to inaccurate results. Additionally, unresolved quantities are often parameterized in state space with empirical relationships or assumed distributions that lack universal applicability. LES is a valuable simulation tool, but its approach to turbulence modeling can introduce unwanted empiricism and make errors difficult to isolate and quantify.

The one-dimensional turbulence model (ODT) functionally reverses the LES approach, modeling large-scale turbulent advection and directly resolving small-scale flow structures, simulating the full range of length and time scales in a single dimension. Because large-scale structures are much easier to study and model than small-scale structures, ODT mitigates or sidesteps many of the subgrid modeling issues that complicate LES. Previous studies show that ODT can attain accuracy comparable to DNS at a fraction of the computational cost \cite{Lignell_2015,Abboud_2015}, making it an attractive tool for simulating turbulent flows. Because the model is one-dimensional, it is limited to homogeneous or boundary layer flows such as jets, wakes, and mixing layers; such flows, however, are extremely common in both nature and turbulence research. ODT's computational efficiency and resolution of a full range of scales make it a valuable tool that complements experimental studies and other simulation tools like DNS and LES.

Early applications of ODT focused on homogenous turbulence, wakes, and mixing layers \cite{Kerstein_1999,Kerstein_2000,Kerstein_2001}. Later extension to variable-density flows and a spatial downstream coordinate system facilitated its growth and application to more complex flows, including combustion in jet flames \cite{Echekki_2001,Hewson_2001,Hewson_2002,Lignell_2012,Punati_2011,Abdelsamie_2017,Lignell_2017, Goshayeshi_2015}, counterflow flames \cite{Jozefik_2015}, wall fires \cite{Monson_2016}, and sooting flames \cite{Lignell_2015,Hewson_2006,Hewson_2009,Lignell_2015b,Ricks_2010}, as well as other particle flows \cite{Sun_2017,Schmidt_2009,Sun_2014,Fistler_2017}. ODT has also served to complement LES through subgrid modeling studies \cite{Cao_2008,Schmidt_2003,Schmidt_2010} and has been applied to various other flow configurations such as double-diffusive interfaces \cite{GonzalezJuez_2011}, Rayleigh-Taylor mixing \cite{GonzalezJuez_2013}, and stratified turbulence \cite{Wunsch_2001}. Most recently, the ODT code was extended to include cylindrical and spherical coordinate systems \cite{Lignell_2018,Klein_2018,Klein_2019}.

During the implementation of the cylindrical and spherical model formulations, the ODT code was drastically overhauled and reorganized, resulting in the current development version of the code. The ODT code presented here is a pared down version of the development code, representing the most fundamental aspects of the ODT model. To this end, the code presented here does not include all of the development code's functionalities. The example cases in Section \ref{sec:examples} represent a good sampling of the ODT code's capabilities as it is presented here.  

%Questions to answer in this section (from SoftwareX template)
%\begin{enumerate}
%	\item What's the scientific background and motivation for this software?
%	\item Why is this important? What problems does it solve?
%	\item How has the software contributed (and/or how will it contribute in the future) to the process of scientific discovery? Cite papers using the software.
%	\item In what experimental setting might someone use this software?
%	\item What related work is there in the literature?
%	\item What algorithms, other code/software, or ideas are used? Cite them. 
%\end{enumerate}

%%%%%%%%%%%%%%%%%%%%%%%%%%%%%%%%%%%%%%%%%%%%%%%%%%%%%%%%%%%%%%%%%%%%%%%%%%%%%%%
\section{Software description}
\label{sec:description}

\subsection{Model description}
\label{sub:model_description}

The ODT model is described in detail in the literature \cite{Kerstein_1999,Kerstein_2001,Ashurst_2005,Lignell_2018,Lignell_2013}; only a brief explanation will be given here. In ODT, turbulent advection is modeled with stochastic processes called eddy events, which punctuate the solution of unsteady, one-dimensional transport equations for mass, momentum, and enthalpy. The ODT code uses a Lagrangian finite-volume formulation for diffusive advancement that includes adaptive mesh refinement \cite{Lignell_2013}. In this approach, mass remains constant inside each grid cell while cell volumes increase or decrease according to dilation. Because the ODT model is one-dimensional, it is limited to homogeneous or boundary-layer flows, such as jets, wakes, and mixing layers; these types of flows, however, are common in nature and central to turbulence research.

Eddy events occur as a Poisson process in accordance with their eddy rates, where a given eddy event of size $l$ and location $x_0$ has an eddy timescale $t$ and an associated eddy rate $1/t$. Three user-defined ODT parameters control the eddy event process: the eddy rate parameter $C$ scales the rate of occurrence of the eddies; the viscous penalty parameter $Z$ suppresses small eddies; and the large eddy suppression parameter $\beta$ constrains eddies such that they do not reach over the elapsed simulation time. 

Eddy events modify domain variables using triplet maps, as illustrated for a cylindrical domain in Figure \ref{fig:tripletmap}. For a region of eddy size $l$, the domain is copied to create three map images; the three images are then placed back to back with the middle image inverted to maintain continuity, and the composite is reapplied to the domain. This process applies to all transported variables on the domain. Applied properly, the triplet map increases scalar gradients and decreases length scales consistent with the application of turbulent eddies in real flows, conserves all quantities and their statistical moments, and maintains continuity in property profiles. Subsequent eddies in the same region will result in a cascade of scales, and eddy rates depend on eddy size and the local kinetic energy such that they follow turbulent cascade scaling laws.  

\begin{figure}
	\centering
	\includegraphics[width=\textwidth]{../figures/tripletmap/tripletmap.png} 
	\caption{Schematic diagram of a cylindrical triplet map, adapted from \cite{Lignell_2018}. Before the triplet map, the domain contains three grid cells of equal volume, while after the triplet map has been applied, the domain contains nine cells. The nine final cells are labeled according to the cells from which they originated and shaded to indicate that three map images were combined to create the final composite.}
	\label{fig:tripletmap}
\end{figure}

%-----------------------------------------------------------------------------------
\subsection{Software Architecture}
\label{sub:architecture}

The ODT code is a relatively self-contained C++ package. The system of nonlinear ODEs is solved using CVODE \cite{Hindmarsh_2020} and user input files are processed with YAML \cite{Beder_2008}, both of which are installed locally during the ODT build process. For reacting flow cases, chemical kinetics and transport are handled by Cantera \cite{Goodwin_2018}, which must be previously installed by the user. 

%Things to put in this section (from SoftwareX template)
%\begin{itemize}
%	\item overview of overall software architecture
%	\item optional: pictorial overview
%	\item implementation details
%\end{itemize}

%%%%%%%%%%%%%%%%%%%%%%%%%%%%%%%%%%%%%%%%%%%%%%%%%%%%%%%%%%%%%%%%%%%%%%%%%%%%%%%
\section{Example Cases}
\label{sec:examples}

\subsection{Pipe Flow}
\label{sub:pipeflow}

% TEXT COPIED FROM CYLINDRICAL ODT PAPER
%We present results for incompressible pipe flow simulations using the temporal, cylindrical ODT formulation. Results for three different friction Reynolds numbers Reτ = 550, 1000, 2000 are compared to DNS results f Kh t l [26](R 550 1000) d Chi t l [6](R 2000) Silti lt from Khoury et al. [26](Reτ = 550, 1000) and Chin et al. [6](Reτ = 2000). Simulation results were dd i i dit f D 2 0 d fl dit f 1 0k −3 F fll d ld i fl it produced using a pipe diameter of D = 2.0 m and flow density of 1.0kg m−3. For fully developed pipe flow, it is possible to estimate the value of a constant mean pressure gradient driving the flow based on the value of the friction velocity, the pipe radius and the density. Friction velocity values of 1 (Reτ = 550, 1000) and 2 ms−1 (R 2000) d d d t llt th dit di i th fl Silti (Reτ = 2000) were assumed and used to calculate the mean pressure gradient driving the flow. Simulation results achieving statistical convergence for the friction velocity were verified afterward as a check on the input parameters. The simulations used initial conditions with constant velocity profiles. The simulations were run to a developed flow state, after which simulation data were gathered until statistical convergence for the root mean square (RMS) velocity difference from the mean profiles occurred. The total normalized run time trun/τpipe = trunu/D was 20200, 25070, and 28140 for Reτ = 550, 1000, and 2000, respectively, where u is the average velocity and D is the pipe diameter.

%The simulations were performed with parameters ofC = 5and Z = 350 for the temporalODTformulation. Additionally, a restriction was imposed on the eddy size range by selecting eddies only up to a maximum normalized size of Le,max/D = 1/3. This restriction limits the eddy size by construction, as opposed to the large-eddy suppression mechanism commonly used in ODT simulations [27,45]. The values of C, Z,and Le,max/D were adjusted to give good agreement of the ODT results compared to the DNS. Schmidt et al. [45]. showed that higher Z results in the buffer-layer being located further from the wall; increasing C results in a lower slope of the mean streamwise velocity in the log-layer; and higher Le,max/D gives a smaller mean streamwise velocity in the wake region.

%Figure 6a shows results of the mean streamwise velocity profiles in wall units for each of the three Reτ considered. The profiles are shifted vertically by 10 and 20 units in the figure for presentation. The Reτ = 550 case shows a comparison of the TMA, TMB, and the hybrid planar/cylindrical approach of the TMB eddy event implementation (PTMB). As seen in the figure, the differences between TMA, TMB, and PTMB are negligible, and so were not considered for the higher Reτ cases. The agreement between the ODT and DNS for the mean velocity is excellent for all three Reτ values.

%Figure 6b shows the streamwise root mean square (RMS) velocity profiles for each Reτ . These profiles are shifted vertically by 2 and 4 units in the figure for presentation. The ODT RMS velocity profiles deviate from the DNS more strongly than the mean velocity profiles, with the ODT value at the peak approximately 20% lower than the DNS. The qualitative shape of the profiles, however, is the same as expected from previous ODT channel flow simulations performed with the planar formulation [33,37]. The small double peak structure of the ODTwas described in [33] and arises from alignment of the triplet map images in the near-wall region. The slight depression between the ODT peaks aligns with the DNS peak, resulting in a larger difference between the RMS profiles than an extrapolation of the surrounding ODT profiles to the peak region would give. As for the mean profiles, the differences between TMA, TMB, and PTMB for the RMS velocity profiles are negligible. Small differences appear only for the RMS profiles near the centerline. We expect this because of the eddy timescale behavior seen in Fig. 5.

\subsection{Non-reacting Jet }
\label{sub:nonreactingjet}

% TEXT COPIED FROM CYLINDRICAL ODT PAPER
%The new cylindrical ODT formulation is demonstrated in a nonreacting round turbulent jet. Results are compared to the experimental data of Hussein et al. [18]. The jet consists of air issuing into air through a 1 in (0.0254 m) diameter duct. The jet exit velocity is 56.2 ms−1 and is well-approximated by a top-hat profile. The reported Reynolds number is 95,500, where Re = Dv0/ν,and D is the jet exit diameter, v0 is the jet exit velocity, and ν is the kinematic viscosity. The ODT simulations use the same diameter and velocity, but a kinematic viscosity of 1.534×10−5 m2 s−1, giving a Reynolds number of 93,056. The initial velocity profile i th ODT ilti i difid t ht fil i hih h bli ttfti f idth δ 0 1D in the ODT simulations is a modified top-hat profile in which a hyperbolic tangent function ofwidth δ = 0.1D is used on either side of the jet to smooth the transition between the jet and the free stream. Here, xc1 and xc2 are the center locations of the tanh transition, −D/2and D/2, respectively. In the spatial formulation ofODT, the streamwise velocity must be positive everywhere on the line due to v in the denominator on the right hand side of the evolution equations (see Sect. 2.4.1). As such, a small vmin = 0.1ms−1 is added uniformly to the velocity profile. v is taken to be the jet velocity of 56.2 ms−1.

%ODT simulations were performed using the TMB triplet map with parameters C = 5.25, βles = 3.5, and Z=400. The value of Z is the same as the spatial simulations in [38] and was not adjusted. The values ofC and βles were adjusted to give good agreement of the jet evolution with the experimental data. Note the very close agreement of the C and Z parameters here to the optimal values used for the pipe flow simulations where C = 5and Z = 350. This illustrates a level of robustness in the ODT parameters between the two configurations and suggests that intermediate values could be successfully applied in both configurations. The pipe flow is sensitive to Z, as noted above, but the jet is not, so Z = 350 would be preferred. Figure 7 shows results of the simulations.

%Here, 1024 independent ODT realizations were performed and results were ensemble averaged. All quantities are normalized consistent with jet similarity scaling. Downstream locations are normalized by the jet diameter D, and radial locations are normalized by (y − y0)where y is the downstream location and y0 = 4D is the virtual origin used in [18]. In the figure, v0 is the jet exit velocity and vcL is the local mean axial centerline velocity. Here, r is used to denote both the experimental radial location and the ODT line position x.

%Figure 7ashows v0/vcL versus y/D; the similarity scaling gives a nominally linear profile where v decays as 1/y. The ODT simulation compares very well with the stationary wire data in [18] after an initial induction period for y/D < 20. The dashed line in the plot is the linear curve fit reported by Hussein et al. Figure 7b shows radial profiles of the mean axial velocity normalized by the local centerline value. Profiles at three axial locations, 50D,70D, and 90D, are shown. The experimental data points shown are the laser Doppler some spread in the profiles at higher r/(y − y0), with the simulation results relaxing toward the experimental values with downstream distance. Figure 7c shows radial profiles of the axial root mean square (RMS) velocity anemometer (LDA) data in [18]. TheODT results show a similarity collapse of the data at r/(y−y0)< 0.1, but normalized by the local centerline velocity at the same three positions. Here again, the profiles tend to relax toward the experimental (LDA) data at higher r/(y − y0) with downstream distance. 

%the quantitative agreement of vrms velocity is generally good, especially for r/(y − y0)> 0.03. At lower values, near the centerline, the ODT vrms increases, whereas the experimental data decrease slightly. We attribute this to the so-called centerline anomaly of the cylindrical ODT formulation due to the geometric stretching associated with the cylindrical triplet maps. This stretching is shown in Figs. 2 and 3.Infact,the motivation for developing TMB was to minimize the degree of this geometric stretching. The stretching effect was present for both maps TMA and TMB, but it is largest near the centerline, where curvature is large. With increasing distance from the centerline, that is, for large distances compared to the eddy size, the geometric stretching effect becomes negligible for both maps, and they both approach the planar limit.

%For comparison to the ODT results with TMB shown in Fig. 7, similar simulations were performed with TMA, and corresponding results are shown in Fig. 8. Here, βles and Z are the same as for the previous case with TMB, but C = 7. Again, we see good agreement for the centerline velocity decay in Fig. 8a. A somewhat better similarity collapse than for TMB occurs in the radial velocity profiles, though they are slightly lower than the experimental data at r/(y − y0)< 0.13 and slightly higher than the experimental data at r/(y − y0)> 0.13. The vrms profiles also show strong similarity scaling. Compared with TMB, however, the simulations show a much higher centerline anomaly, and the vrms profile departs from the experiments much further from the centerline and has a higher rise in the region r/(y − y0)< 0.05. 

%It is worth noting that the vrms profile depends somewhat on the ODT parameters selected. Hewson and Kerstein [15] demonstrated that mean centerline values of mixture fraction were insensitive to C and βles provided their ratio was kept constant. In these simulations, however, higher values of RMS fluctuations were observed at lower values of βles. Figure 9 shows results for TMA similar to those in Fig. 8 with varying C and βles. Each of the three cases has C/βles = 2, with C = 7, 5, and 3. Plots (b) and (c) show axial location

%The centerline velocity decay and radial profiles show good agreement, as with the previous cases, with a slight improvement in the radial velocity profile. The vrms profiles are significantly improved along the centerline compared to TMA and TMB. The centerline anomaly is not completely removed, however, since implemented eddies are still subject to the geometric distortion of the TMB map, whose influence is felt by the τ−1 calculation for subsequent eddy events. e


\subsection{Jet Flame}
\label{sub:jetflame}

% TEXT COPIED FROM CYLINDRICAL ODT PAPER
%The newcylindrical ODTformulation is demonstrated in a round, reacting, turbulent jet flame.Results are compared to the experimental DLR-Aflame ofMeier et al. [19,36]. Heat release and temperature- and compositiondependent transport and thermodynamic properties are key characteristics of this flow. This canonical flame configuration has been used extensively to study and validate turbulent combustion models. Pitsch [41] studied differential diffusion effects in this flame using a classical unsteady flamelet model. Lindstedt and Ozarivsky [35] investigated it using a joint PDF approach, while Wang and Pope [52] used an LES/PDF model combined with a flamelet/progress-variable (FPV) model. Fairweather and Woolley [10] studied several chemical mechanisms using the DLR flame and a first-order conditional moment closure (CMC) model, and Lee and Choi [29,30] studied NO emissions using an Eulerian particle flamelet model. These studies consisted of computational fluid dynamics (CFD) simulations with subgrid modeling of the turbulent combustion process. In contrast, ODT is a low-dimensional representation of the entire flow, not just the subgrid scales. Previous ODT studies of turbulent jet flames have employed the temporal planar formulation, so we expect that the model may improve with a spatial cylindrical formulation that more closely matches the experimental configuration.
%The DLR-A fuel stream is mixture of 22.1% CH4, 33.2% H2, and 44.7% N2 (by volume) that issues into dry air. The fuel stream exits a nozzle with an inner diameter of 8 mm at a mean exit velocity of 42.2 ms−1. The coflowing air stream issues from a nozzle of diameter 140 mm at a velocity of 0.3 ms−1. The reported Reynolds number is 15,200.
%The ODT simulation uses these diameters and the experimentally reported velocity profile at the jet exit. In the nonreacting case, a small velocity was added uniformly to the profile, no velocity addition is needed here because of the slow-moving coflow air stream present alongside the reacting jet. The ODT simulation includes the buoyant source term but not the streamwise pressure gradient in Eq. (25). The fuel was diluted with N2 in the experimental flame to minimize radiative heat losses, and radiation is ignored in the simulation. This flame has a low Reynolds number, and the combustion chemistry proceeds quickly. The ODT simulation transports the following species: O2, N2, CH4, H2, H2O, CO2. Reactions are assumed to proceed to products of complete combustion:
%H2 + 1/2O2 → H2O, CH4 + 2O2 → CO2 + 2H2O,
%(38) (39)
%and simple, fast reaction rates are applied. These assumptions are not reasonable for the DLR-A flame. The primary purpose of these simulations is to illustrate ODT in a reacting jet with variable properties and heat release.
%Z = 400. The values of C and βles were adjusted to give good agreement of the jet evolution with the experimental data, while the value of Z was taken as the same value as for the jet in Sect. 3.2. Results of ODT simulations were performed using the TMB triplet map with parameters C = 20, βles = 17, and 400 Th l f C d β djtd t i d t f th jt lti ith th
%the simulation are presented in Figs. 11 and 12. One thousand independent flow realizations were performed, and the results ensemble averaged. Downstream distance, y, and radial position, r, are normalized by the jet diameter D.


%Present the major functionalities of the software. 
%\begin{itemize}
%	\item simulating turbulent flow cases, reacting or nonreacting flows
%	\item can simulate laminar flow cases too
%	\item does these things a whole lot faster than other methods do (LES, DNS especially)
%	\item testing LES subgrid modeling assumptions
%	\item simulating cases that DNS can't get to because needed simulation length is too long (i.e. late-flame phenomena)
%\end{itemize}

%%%%%%%%%%%%%%%%%%%%%%%%%%%%%%%%%%%%%%%%%%%%%%%%%%%%%%%%%%%%%%%%%%%%%%%%%%%%%%%
\section{Impact}
\label{sec:impact}

Questions to answer in this section (from SoftwareX template)
\begin{enumerate}
	\item How can new research questions be pursued with this software?
		\begin{itemize}
			\item possibility of parametric studies (much harder with DNS/LES/RANS)
			\item study of late-flame soot and radiation interactions, soot emissions as smoke
			\item comparative radiation model studies?
		\end{itemize}
	\item How does the software improve pursuit of existing research questions?
		\begin{itemize}
			\item late-flame behavior becomes easier to study
			\item validation of LES subgrid models
			\item soot stuff, especially late in the flame (because soot moves slowly compared to gas species and therefore short simulation times like in DNS aren't enough to study it effectively)
		\end{itemize}
	\item How does the software change the daily practice of its users?
		\begin{itemize}
			\item cases take hours or days rather than weeks using supercomputer resources
			\item test cases can be run on local computers (unlike something like DNS) and as background tasks without disrupting other tasks
			\item ODT as a tool complements other approaches, can cover blind spots and be used in validation
		\end{itemize}
	\item How widespread is the software? Who uses it? (Within and outside of intended research area and/or group.)
		\begin{itemize}
			\item BYU group
			\item JCH at Sandia
			\item Chalmers group in Sweden (Marco Fistler, etc.)
			\item German university group (Heiko Schmidt, Juan Media, Marten Klein, etc.)
			\item TO DO: find other groups who have used or currently use ODT
		\end{itemize}
	\item How is the software used in commercial settings (if any)? Has it led to creation of spin-off companies?
		\begin{itemize}
			\item No commercial use (I think). 
		\end{itemize}
\end{enumerate}

%%%%%%%%%%%%%%%%%%%%%%%%%%%%%%%%%%%%%%%%%%%%%%%%%%%%%%%%%%%%%%%%%%%%%%%%%%%%%%%
\section{Conclusion}
\label{sec:conclusion}

Write this part next to last
%Set out the conclusion of this original software publication.

%%%%%%%%%%%%%%%%%%%%%%%%%%%%%%%%%%%%%%%%%%%%%%%%%%%%%%%%%%%%%%%%%%%%%%%%%%%%%%%
\section{Conflict of Interest}
%Please select the appropriate text:
%
%Potential conflict of interest exists:
%We wish to draw the attention of the Editor to the following facts, which may be considered as potential conflicts of interest, and to significant financial contributions to this work. The nature of potential conflict of interest is described below: [Describe conflict of interest]

%No conflict of interest exists:
We wish to confirm that there are no known conflicts of interest associated with this publication and there has been no significant financial support for this work that could have influenced its outcome.

%%%%%%%%%%%%%%%%%%%%%%%%%%%%%%%%%%%%%%%%%%%%%%%%%%%%%%%%%%%%%%%%%%%%%%%%%%%%%%%
\section*{Acknowledgements}
\label{acknoledgements}

This work is supported in part by the National Science Foundation under Grant No. CBET-1403403.

%Special thanks to Alan R. Kerstein for his significant contributions to model development. Additional thanks to Heiko Schmidt, Juan Medina, and Marten Klein of Brandenburg University of Technology Cottbus-Senftenberg, Michael Oevermann and Marco Fistler of Chalmers University of Technology, and Vladimir P. Solovjov of Brigham Young University.

%%%%%%%%%%%%%%%%%%%%%%%%%%%%%%%%%%%%%%%%%%%%%%%%%%%%%%%%%%%%%%%%%%%%%%%%%%%%%%%
%% The Appendices part is started with the command \appendix;
%% appendix sections are then done as normal sections
%% \appendix

%% \section{}
%% \label{}

%%%%%%%%%%%%%%%%%%%%%%%%%%%%%%%%%%%%%%%%%%%%%%%%%%%%%%%%%%%%%%%%%%%%%%%%%%%%%%%
%% References:

\bibliographystyle{elsarticle-num} 
\bibliography{references} 

%%%%%%%%%%%%%%%%%%%%%%%%%%%%%%%%%%%%%%%%%%%%%%%%%%%%%%%%%%%%%%%%%%%%%%%%%%%%%%%
\section*{Current executable software version}
\label{software_version}

Ancillary data table required for sub version of the executable software: (x.1, x.2 etc.) kindly replace examples in right column with the correct information about your executables, and leave the left column as it is.

\begin{table}[!h]
\begin{tabular}{|l|p{6.5cm}|p{6.5cm}|}
\hline
\textbf{Nr.} & \textbf{(Executable) software metadata description} & \textbf{Please fill in this column} \\
\hline
S1 & Current software version & 2.1 \\
\hline
S2 & Permanent link to executables of this version  & For example: $https://github.com/combogenomics/$ $DuctApe/releases/tag/DuctApe-0.16.4$ \\
\hline
S3 & Legal Software License & MIT \\
\hline
S4 & Computing platforms/Operating Systems & Linux, OS X, Microsoft Windows\\
\hline
S5 & Installation requirements \& dependencies & CMake 3.12+, Cantera, Git, Doxygen (optional) \\
\hline
S6 & If available, link to user manual - if formally published include a reference to the publication in the reference list & For example: $http://mozart.github.io/documentation/$ \\
\hline
S7 & Support email for questions & davidlignell@byu.edu \\
\hline
\end{tabular}
\caption{Software metadata (optional)}
\end{table}

\end{document}
\endinput
